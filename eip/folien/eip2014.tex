\documentclass{hswbeamer}

\title[EiP]{Einführung in die Programmierung}
\subtitle[WS 2014/15]{Wintersemester 2014/15}
\author[\Copyright{Uwe L\"{a}mmel, Thomas Jonitz}]{Prof. Dr.-Ing Uwe L\"{a}mmel \and Thomas Jonitz (B.Sc.)}
\date{Wismar, \today}
%\logo{\includegraphics[height=1cm]{fischer.png}}
\begin{document}



\begin{frame}
    \begin{quote}
        \glqq Saying Java is nice, because it works on all OSes ist like saying anal sex is nice, because it works on all genders.\grqq -- Icematix (c-plusplus.de)
    \end{quote}
\end{frame}


\begin{frame}{Vorstellung}
    \begin{description}
    \item[Name:]Thomas Jonitz ($\ast$ 1987 in Stralsund)
    \item[2007:]Abitur (Mathematik, Informatik)
    \item[2009:]Industrietechnologe (Datentechnik)
    \item[2013:]Bachelor of Science (Wirtschaftsinfromatik)\\Masterstudiengang WI (IT-SysArch)
    \item[Trivia:]Regelmäßige Tutorien und Seminare in \Quote{Einführung in die Programmierung} und \Quote{Anwendungsprogrammierung} seit 2009.
    \end{description}   
\end{frame}

\begin{frame}
    \maketitle
\end{frame}

\begin{frame}{Inhaltsverzeichnis}{Agenda}
    \tableofcontents
\end{frame}

\section{Einführung}\label{sec:0-einfuehrung}
\begin{frame}{Einführung}
\end{frame}


\begin{frame}{Ablauf der Lehrveranstaltung}
\end{frame}

\begin{frame}{Motto}
\end{frame}

\begin{frame}{Problem}
\end{frame}

\begin{frame}{Literaturempfehlung}
\end{frame}

\section{Klassendefinition und Objekte}
\begin{frame}{Sample}
\end{frame}

\section{Anweisungen und Datentypen}
\begin{frame}{Sample}
\end{frame}

\section{Datensammlungen}
\begin{frame}{Sample}
\end{frame}

\section{Container-Klassen}
\begin{frame}{Sample}
\end{frame}

\section{Algorithmen}
\begin{frame}{Sample}
\end{frame}

\section{Vererbung}
\begin{frame}{Sample}
\end{frame}

\section{Statische Methoden}
\begin{frame}{Sample}
\end{frame}

\section{Zusammenfassung}
\begin{frame}{Sample}
\end{frame}



\end{document}