\TitleFrame{Programmiersprache Java}

\begin{frame}{Programmiersprache Java}{Quelltext $\rightarrow$ Programm}
    \begin{block}{C-Programm}
    Quelltext $\rightarrow$ Compiler $\rightarrow$ Maschinencode $\rightarrow$ OS
    \end{block}
    \begin{block}{Java-Programm}
    Quelltext $\rightarrow$ Compiler $\rightarrow$ Bytecode $\rightarrow$ JVM $\rightarrow$ Maschinencode $\rightarrow$ OS
    \end{block}
\end{frame}

\begin{frame}{Programmiersprache Java}{Architektur}
    \begin{itemize}
        \item[JVM:]
        \item[JRE:]
        \item[JDK:]
    \end{itemize}
\end{frame}

\begin{frame}{Programmiersprache Java}{Java Virtual Machine}
    \ldots
\end{frame}

\begin{frame}{Programmiersprache Java}{Java Virtual Machine}
    \ldots
\end{frame}

\begin{frame}{Programmiersprache Java}{Paradigmen}
    \begin{itemize}
        \item Java ist objektorientiert.
        \item Namen werden nach dem \Quote{CamelCase}-Prinzip vergeben.
        \item Klassen und Datenstrukturen beginnen mit Großbuchstaben.
        \item Variablen und Methoden beginnen mit Kleinbuchstaben.
        \item Eine Datei, eine Klasse.
        \item Die Datei muss exakt so benannt sein, wie die darin beschriebene Klasse
    \end{itemize}
\end{frame}

\begin{frame}{Programmiersprache Java}{}
    \ldots
\end{frame}
