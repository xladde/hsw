\TitleFrame{Java-Programmierung}

\subsection{Operatoren}
\begin{frame}{Java-Programmierung}{Operatoren}
    \begin{block}{arithmetische Operatoren}
        \centering{$\texttt{+, ++, --, +=, -=}$}
        \centering{$\texttt{*, /, \%, *=, /=, \%=}$}
    \end{block}
    \begin{block}{logische und vergleichende Operatoren}
        \centering{$\&\&, \mid\mid, <, <=, >, >=, !, !=$}
    \end{block}
\end{frame}

\subsection{Datentypen, Variablen, Anweisungen}
\begin{frame}[shrink]{Java-Programmierung}{Datentypen}
    \begin{block}{\Quote{primitive} Datentypen}
        \centering{\texttt{int, float, boolean, char, double}\ldots}
    \end{block}
    \begin{block}{Datenstrukturen}
        \centering{\texttt{String, EigeneKlassen, Interfaces}\ldots}
    \end{block}
    \begin{exampleblock}{Hinweise}
        \begin{itemize}
            \item Datenstrukturen beginnen \textbf{immer} mit Großbuchstaben!
            \item Datenstrukturen können der Einfachheit halber als \Quote{Datentypen} angesehen werden. 
        \end{itemize}
    \end{exampleblock}
\end{frame}

\begin{frame}[shrink]{Java-Programmierung}{Variablen}
    \begin{block}{Variablen-Deklaration/-Definition}
        \Code{code/var-deklaration.java}{Allgemeines Prinzip}
    \end{block}
    \begin{block}{Beispiel}
        \Code{code/var-deklaration-ex1.java}{Beispiel}
    \end{block}
\end{frame}
\begin{frame}{Java-Programmierung}{Variablen}
    \begin{block}{Deklaration, Definition und Instanzierung}
        \begin{description}
            \item[Deklaration]legt fest, dass eine Variable mit diesem Namen im vorgegebenen Gültigkeitsbereich existiert.
            \item[Definition/Instanzierung]legt zudem fest, welchen Wert/Zustand die Variable annimmt; es wird ein Objekt (Instanz) erzeugt (speziell OOP) 
        \end{description}
    \end{block}
    \begin{block}{Klassen, Objekte und Instanzen}
        \begin{description}
            \item[Klasse:]\Quote{Vorlage} für eine Instanz (ausführbare Klassen, Bibliotheksklassen, Bauplanklassen)
            \item[Objekt/Instanz:]Das eindeutige und konkrete Ausprägung einer Klasse zur Laufzeit des Programms
        \end{description}
    \end{block}
\end{frame}

\begin{frame}{Java-Programmierung}{Datenbehälter}
    \begin{block}{Datenbehälter}
        \begin{itemize}
            \item Arrays (Liste von Daten selben Typs)
            \item Array-Lists (Datenstruktur zur Verwaltung von Daten selben Typs)
        \end{itemize}
    \end{block}
\end{frame}


\subsection{Fallunterscheidung}
\begin{frame}{Java-Programmierung}{Fallunterscheidung}
    \begin{itemize}
        \item\texttt{if-else}-Unterscheidung
        \item\texttt{switch-case}-Unterscheidung
    \end{itemize}
\end{frame}


\subsection{Schleifen}
\begin{frame}{Java-Programmierung}{Schleifen}
    \begin{block}{Kopfgesteuerte Schleifen}
        \texttt{for}-Schleife, \texttt{while}-Schleife
    \end{block}
    \begin{block}{Fußgesteuerte Schleifen}
        \texttt{do-while}-Schleife
    \end{block}
    \begin{block}{sonstige Schleifen}
        \texttt{for-each}-Schleife
    \end{block}
\end{frame}

\subsection{Methoden}
\begin{frame}{Java-Programmierung}{Methoden}
    \Code{code/javamethod.java}{Java-Methoden}{}
\end{frame}