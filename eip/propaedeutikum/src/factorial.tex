\TitleFrame{Die Fakultätsfunktion}

%\subsection{Iterative Umsetzung}
\begin{frame}[shrink]{Fakultätsfunktion}{Iteratives Bildungsgesetz}
    \begin{block}{Beispiel}
        $5! = 5\times 4\times 3\times 2\times 1 = 120$
    \end{block}
    \begin{block}{Bedingung}
        \centering{$n\in\mathbb{N}$}
    \end{block}
    \begin{block}{iteratives Bildungsgesetz}
        \centering{$n! = 1\times 2\times\ldots\times n$}
        \[n! = \prod_{k=1}^{n}k\]
    \end{block}
\end{frame}

%\begin{frame}[shrink]{Fakultätsfunktion}{Iterativer Algorithmus}
%\textbf{Typ} $zahl\in\mathbb{N}$\\
%\textbf{Funktion} \textit{fakultaet(\textbf{Variable} $n: zahl$)}: $zahl$\\
%\quad\textbf{Variable} $i$, $ergebnis$: $zahl$\\
%\quad$i = 1$\\
%\quad$ergebnis = 1$\\
%\quad \textbf{solange} $i\leq n$:\\
%\quad\quad$ergebnis = ergebnis * i$\\
%\quad\quad$i = i + 1$\\
%\quad\textbf{return} $ergebnis$
%\end{frame}

\begin{frame}{Fakultätsfunktion}{Iterative Implementierung in Java}
    \Code{./code/factorial-iterative.java}{iterative Implementierung}{}
\end{frame}


%\subsection{Rekursive Umsetzung}
\begin{frame}[shrink]{Fakultätsfunktion}{Rekursives Bildungsgesetz}
    \begin{block}{Beispiel}
        $4! = 4\times 3\times 2\times 1\times = 24$\\
        $5\times 24 = 120\rightarrow 5! = 5\times 4!$\\
    \end{block}
    \begin{block}{Bedingung}
        \centering{$n\in\mathbb{N}$}
    \end{block}
    \begin{block}{rekursives Bildungsgesetz:}
        \[
            n! = 
                \begin{cases}
                    1       &\quad\text{für }n = 0\\
                    n(n-1)! &\quad\text{für }n > 0
                \end{cases}
        \]
    \end{block}
\end{frame}

%\begin{frame}{Fakultätsfunktion}{Rekursiver Algorithmus}
%    \ldots
%\end{frame}


\begin{frame}{Fakultätsfunktion}{Rekursive Implementierung in Java}
    \Code{./code/factorial-recursive.java}{rekursive Implementierung}{}
\end{frame}

%\subsection{Beispiele}
\begin{frame}{Fakultätsfunktion}{Variante 1}
    \Code{./code/ex1/Factorial.java}{Factorial.java}{}
\end{frame}
\begin{frame}{Fakultätsfunktion}{Variante 1}
    \Code{./code/ex1/Test.java}{Test.java}{}
\end{frame}