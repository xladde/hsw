%
% KLASSEN UND METHODEN
%
\TitleFrame{Klassen und OOP}
\begin{frame}{Klassen und OOP}{Allgemeiner Aufbau}
    \Code{code/javaclass.java}{allgemeine Java-Klasse}{}
\end{frame}

\subsection{Gültigkeitsbereiche (Scopes)}
\begin{frame}[shrink]{Klassen und OOP}{Gültigkeitsbereiche}
Gültigkeitsbereiche (häufig auch \Quote{Scopes} genannt), definieren den Bereich, in dem die so gekennzeichneten Objekte bekannt sind.

Es werden drei Gültigkeitsbereiche unterschieden:
    \begin{description}
        \item[private (-):] Das Attribut ist nur innerhalb dieses Objektes gültig. 
        \item[protected (\#):] Das Attribut ist nur innerhalb dieser Vererbungshierarchie gültig.
        \item[public (+):] Das Attribut ist überall gültig. 
    \end{description}
Klassen, Variablen und Methoden können mit einem Scope gekennzeichnet werden.
\end{frame}

\subsection{Vererbung}
%\begin{frame}{Klassen und OOP}{Vererbung}
%    \Code{code/javaclass.java}{Java-Klassen}{Vererbung}
%\end{frame}