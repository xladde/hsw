\documentclass{hswarticle}

\author{Thomas Jonitz}
\title{Dokumentation  \LaTeX -Klasse \texttt{hswarticle}}
\date{Wismar, \today}

\makeindex
\makenomenclature
\begin{document}
    \maketitle
    \vfill
    \begin{center}
        \includegraphics[scale=0.5]{hsw-all-dark.eps}
    \end{center}
    \vfill
    \begin{abstract}
        Nachfolgende Dokumentation beschreibt die \LaTeX -Klasse \texttt{hswarticle}, die als Grundlage für wissenschaftliche Dokumente an der Hochschule Wismar verwendet werden kann. Sie folgt grundlegenden Formatierungsrichtlinien und orientiert sich am Corparate Design der Hochschule.
    \end{abstract}
    \clearpage
    
    \addcontentsline{toc}{section}{Inhaltsverzeichnis}
    \tableofcontents
    
    
    \clearpage
    
    \section{Einstieg und Verwendung}
        Die Klasse \texttt{hswarticle} muss sich im Verzeichnis des zu erstellenden Dokuments befinden. Geladen wird sie mit nachfolgender Anweisung.
        \begin{lstlisting}[caption={Laden der Klasse}]
\documentclass{hswarticle}
        \end{lstlisting}        
        Danach folgen bereits die Angaben zu Autor, Titel und Datum. Die Grundkonfiguration muss in der Regel nicht verändert werden. Die Seitenmaße sind mit dem \texttt{geometry}-Paket fix gesetzt, können in der Klasse aber angepasst werden.
        \begin{lstlisting}[caption={Konfiguration der Seitenränder \texttt{hswarticle.cls}}]
\RequirePackage{geometry}
\geometry{
    inner=3cm,
    outer=3cm,
    top=3cm,
    bottom=3cm
}
        \end{lstlisting}         
        Ein Dokument kann also entsprechend folgenden Aufbau haben:
        \begin{lstlisting}[caption={Grundkonfiguration}] 
\documentclass{hswarticle}
\title{Ein \LaTeX -Dokument ohne Inhalt}
\author{Max Mustermann}
\date{\today}
\begin{document}
    
\end{document}
        \end{lstlisting}
        Innerhalb der \texttt{document}-Umgebung kann nun der eigentliche Content mit den üblichen \LaTeX -Anweisungen erstellt werden (\texttt{section}, \texttt{subsection}, \verb+\maketitle+, usw.).
             
    
    \section{Verzeichnisse und Sonderseiten}
        \subsection{Titelseite}
            \lstinputlisting[language=TeX,caption={Titelseite siehe Abbildung \ref{fig:titlepage}}]{img/titlepage.tex}
            \begin{figure}[p]
                \begin{center}
                \fbox{\includegraphics[scale=0.6]{img/titlepage.pdf}}
                \caption{Beispiel-Titelseite}
                \label{fig:titlepage}
                \end{center}
            \end{figure}
        
        
        \subsection{Nomenklatur}
            \lstinputlisting[language=TeX,caption={Nomenklatur siehe Abbildung \ref{fig:nomenclature}}]{img/nomenclature.tex}
            \begin{figure}[p]
                \begin{center}
                \fbox{\includegraphics[scale=0.6]{img/nomenclature.pdf}}
                \caption{Beispiel-Nomenklatur}
                \label{fig:nomenclature}
                \end{center}
            \end{figure}
        
        \subsection{Literaturverzeichnis}
        \lstinputlisting[language=TeX,caption={Bib\TeX -Datei}]{img/literature.bib}
        \lstinputlisting[language=TeX,caption={Litraturverzeichnis siehe Abbildung \ref{fig:literature}}]{img/literature.tex}
            \begin{figure}[p]
                \begin{center}
                \fbox{\includegraphics[scale=0.6]{img/literature.pdf}}
                \caption{Beispiel-Literaturverzeichniss (Stil: \texttt{alphadin})}
                \label{fig:literature}
                \end{center}
            \end{figure}
            \begin{figure}[p]
                \begin{center}
                \fbox{\includegraphics[scale=0.6]{img/literature-plain.pdf}}
                \caption{Beispiel-Literaturverzeichnis (Stil: \texttt{plain})}
                \label{fig:literature}
                \end{center}
            \end{figure}
        Alternativ wird auch der Stil \texttt{plain} empfohlen.
        
        
        \subsection{Anhang}
        
        
    \section{Farbgebung und Logos}
    
    \section{Inhalte erstellen}
        \subsection{Zitate}
        \subsection{Hervorhebungen}
        \subsection{Abbildungen}
        \subsection{Tabellen}
        \subsection{Mathematik-Umgebung}
        \subsection{Algorithmen}
        \subsection{Quelltexte}
    \section{Zusätzliche Befehle}
    \section{Sonderzeichen und zusätzliche Symbole}
    
    \clearpage
    \appendix
    \pagenumbering{Roman}
    \addcontentsline{toc}{section}{Abbildungsverzeichnis}
    \listoffigures
    \addcontentsline{toc}{section}{Tabellenverzeichnis}
    \listoftables
    
\end{document}